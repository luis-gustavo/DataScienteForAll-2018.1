\documentclass[]{article}
\usepackage{lmodern}
\usepackage{amssymb,amsmath}
\usepackage{ifxetex,ifluatex}
\usepackage{fixltx2e} % provides \textsubscript
\ifnum 0\ifxetex 1\fi\ifluatex 1\fi=0 % if pdftex
  \usepackage[T1]{fontenc}
  \usepackage[utf8]{inputenc}
\else % if luatex or xelatex
  \ifxetex
    \usepackage{mathspec}
  \else
    \usepackage{fontspec}
  \fi
  \defaultfontfeatures{Ligatures=TeX,Scale=MatchLowercase}
\fi
% use upquote if available, for straight quotes in verbatim environments
\IfFileExists{upquote.sty}{\usepackage{upquote}}{}
% use microtype if available
\IfFileExists{microtype.sty}{%
\usepackage{microtype}
\UseMicrotypeSet[protrusion]{basicmath} % disable protrusion for tt fonts
}{}
\usepackage[margin=1in]{geometry}
\usepackage{hyperref}
\hypersetup{unicode=true,
            pdftitle={Untitled},
            pdfauthor={Luis Gustavo},
            pdfborder={0 0 0},
            breaklinks=true}
\urlstyle{same}  % don't use monospace font for urls
\usepackage{color}
\usepackage{fancyvrb}
\newcommand{\VerbBar}{|}
\newcommand{\VERB}{\Verb[commandchars=\\\{\}]}
\DefineVerbatimEnvironment{Highlighting}{Verbatim}{commandchars=\\\{\}}
% Add ',fontsize=\small' for more characters per line
\usepackage{framed}
\definecolor{shadecolor}{RGB}{248,248,248}
\newenvironment{Shaded}{\begin{snugshade}}{\end{snugshade}}
\newcommand{\KeywordTok}[1]{\textcolor[rgb]{0.13,0.29,0.53}{\textbf{#1}}}
\newcommand{\DataTypeTok}[1]{\textcolor[rgb]{0.13,0.29,0.53}{#1}}
\newcommand{\DecValTok}[1]{\textcolor[rgb]{0.00,0.00,0.81}{#1}}
\newcommand{\BaseNTok}[1]{\textcolor[rgb]{0.00,0.00,0.81}{#1}}
\newcommand{\FloatTok}[1]{\textcolor[rgb]{0.00,0.00,0.81}{#1}}
\newcommand{\ConstantTok}[1]{\textcolor[rgb]{0.00,0.00,0.00}{#1}}
\newcommand{\CharTok}[1]{\textcolor[rgb]{0.31,0.60,0.02}{#1}}
\newcommand{\SpecialCharTok}[1]{\textcolor[rgb]{0.00,0.00,0.00}{#1}}
\newcommand{\StringTok}[1]{\textcolor[rgb]{0.31,0.60,0.02}{#1}}
\newcommand{\VerbatimStringTok}[1]{\textcolor[rgb]{0.31,0.60,0.02}{#1}}
\newcommand{\SpecialStringTok}[1]{\textcolor[rgb]{0.31,0.60,0.02}{#1}}
\newcommand{\ImportTok}[1]{#1}
\newcommand{\CommentTok}[1]{\textcolor[rgb]{0.56,0.35,0.01}{\textit{#1}}}
\newcommand{\DocumentationTok}[1]{\textcolor[rgb]{0.56,0.35,0.01}{\textbf{\textit{#1}}}}
\newcommand{\AnnotationTok}[1]{\textcolor[rgb]{0.56,0.35,0.01}{\textbf{\textit{#1}}}}
\newcommand{\CommentVarTok}[1]{\textcolor[rgb]{0.56,0.35,0.01}{\textbf{\textit{#1}}}}
\newcommand{\OtherTok}[1]{\textcolor[rgb]{0.56,0.35,0.01}{#1}}
\newcommand{\FunctionTok}[1]{\textcolor[rgb]{0.00,0.00,0.00}{#1}}
\newcommand{\VariableTok}[1]{\textcolor[rgb]{0.00,0.00,0.00}{#1}}
\newcommand{\ControlFlowTok}[1]{\textcolor[rgb]{0.13,0.29,0.53}{\textbf{#1}}}
\newcommand{\OperatorTok}[1]{\textcolor[rgb]{0.81,0.36,0.00}{\textbf{#1}}}
\newcommand{\BuiltInTok}[1]{#1}
\newcommand{\ExtensionTok}[1]{#1}
\newcommand{\PreprocessorTok}[1]{\textcolor[rgb]{0.56,0.35,0.01}{\textit{#1}}}
\newcommand{\AttributeTok}[1]{\textcolor[rgb]{0.77,0.63,0.00}{#1}}
\newcommand{\RegionMarkerTok}[1]{#1}
\newcommand{\InformationTok}[1]{\textcolor[rgb]{0.56,0.35,0.01}{\textbf{\textit{#1}}}}
\newcommand{\WarningTok}[1]{\textcolor[rgb]{0.56,0.35,0.01}{\textbf{\textit{#1}}}}
\newcommand{\AlertTok}[1]{\textcolor[rgb]{0.94,0.16,0.16}{#1}}
\newcommand{\ErrorTok}[1]{\textcolor[rgb]{0.64,0.00,0.00}{\textbf{#1}}}
\newcommand{\NormalTok}[1]{#1}
\usepackage{graphicx,grffile}
\makeatletter
\def\maxwidth{\ifdim\Gin@nat@width>\linewidth\linewidth\else\Gin@nat@width\fi}
\def\maxheight{\ifdim\Gin@nat@height>\textheight\textheight\else\Gin@nat@height\fi}
\makeatother
% Scale images if necessary, so that they will not overflow the page
% margins by default, and it is still possible to overwrite the defaults
% using explicit options in \includegraphics[width, height, ...]{}
\setkeys{Gin}{width=\maxwidth,height=\maxheight,keepaspectratio}
\IfFileExists{parskip.sty}{%
\usepackage{parskip}
}{% else
\setlength{\parindent}{0pt}
\setlength{\parskip}{6pt plus 2pt minus 1pt}
}
\setlength{\emergencystretch}{3em}  % prevent overfull lines
\providecommand{\tightlist}{%
  \setlength{\itemsep}{0pt}\setlength{\parskip}{0pt}}
\setcounter{secnumdepth}{0}
% Redefines (sub)paragraphs to behave more like sections
\ifx\paragraph\undefined\else
\let\oldparagraph\paragraph
\renewcommand{\paragraph}[1]{\oldparagraph{#1}\mbox{}}
\fi
\ifx\subparagraph\undefined\else
\let\oldsubparagraph\subparagraph
\renewcommand{\subparagraph}[1]{\oldsubparagraph{#1}\mbox{}}
\fi

%%% Use protect on footnotes to avoid problems with footnotes in titles
\let\rmarkdownfootnote\footnote%
\def\footnote{\protect\rmarkdownfootnote}

%%% Change title format to be more compact
\usepackage{titling}

% Create subtitle command for use in maketitle
\newcommand{\subtitle}[1]{
  \posttitle{
    \begin{center}\large#1\end{center}
    }
}

\setlength{\droptitle}{-2em}
  \title{Untitled}
  \pretitle{\vspace{\droptitle}\centering\huge}
  \posttitle{\par}
  \author{Luis Gustavo}
  \preauthor{\centering\large\emph}
  \postauthor{\par}
  \predate{\centering\large\emph}
  \postdate{\par}
  \date{4/8/2018}


\begin{document}
\maketitle

Disciplina Ciência de Dados Aplicada e Ciência de Dados para Todos
Relatório 1 -- Importação e Limpeza de Dados Autor: XXXXXXX Data: XXXXXX

\subsection{1. Introdução}\label{introducao}

\subsection{2. Metodologia}\label{metodologia}

Para realizar a análise dos dados foi utilizado o programa RStudio e os
pacotes jsonlite e dplyr para importação e limpeza dos dados.

Para melhor entendimento e visualização dos dados foram utilizados as
funções \emph{str(), glimpse() e summary()}, que proporcionaram uma
visão geral dos dados, e também as funções \emph{length(), class(),
names()} para entender melhor a estrutura dos dados que estavam em
formato json.

Por fim, foram utilizadas as funções \emph{lapply() e sapply()} para a
criação de novas estruturas e a formação de subconjuntos

\subsection{3. Resultados}\label{resultados}

Foram importados quatro arquivos, todos importados através do método
\emph{read\_json()}: * \textbf{Perfis Professores UnB - JSON} :
denominado como \emph{unb\_perfis\_json} * \textbf{Publicações
Professores UnB - JSON} : denominado como
\emph{unb\_relatorio\_plublicao\_json} * \textbf{Orientações Professores
UnB - JSON} : denominado como \emph{unb\_relatorio\_orientacao\_json} *
\textbf{Teses e Dissertações da UnB - JSON} : denominado como
\emph{unb\_teses\_dissertacoes\_json}

O primeiro arquivo, \emph{unb\_perfis\_json}, contém 1592 observações,
que pode ser visto através do método: \emph{length(unb\_perfis\_json)}.
Por meio da função \emph{names(unb\_perfis\_json)} pode-se perceber que
cada observação tem uma string composta de 16 caracteres numéricos que
idenfica a observação. E, por meio da função: \emph{summary()}, pode-se
observar que cada observação é divida em 7 atributos principais:
\textbf{nome, resumo\_cv, areas\_de\_atuacao, endereco\_profissional,
producao\_bibliografica, orientacoes\_academicas e senioridade}.

O segundo arquivo, \emph{unb\_relatorio\_plublicao\_json}, temos os
dados referentes a publicações de professores em: \textbf{periódicos,
livros e capítulos de livros, texto em jornais e artigos aceitos}, isso
pode ser visto utilizando a função \emph{names()}. Para os artigos
aceitos pode-se perceber que o número vem crescendo desde 2012.

\begin{Shaded}
\begin{Highlighting}[]
\KeywordTok{library}\NormalTok{(jsonlite)}
\NormalTok{unb_relatorio_plublicao_json <-}\StringTok{ }\KeywordTok{read_json}\NormalTok{(}\StringTok{'unb.relatorioProducaoBibiografica.json'}\NormalTok{)}
\NormalTok{artigos_aceitos <-}\StringTok{ }\KeywordTok{lapply}\NormalTok{(unb_relatorio_plublicao_json}\OperatorTok{$}\NormalTok{ARTIGO_ACEITO, length)}
\NormalTok{anos <-}\StringTok{ }\KeywordTok{names}\NormalTok{(artigos_aceitos)}
\NormalTok{numero_de_artigos_aceitos <-}\StringTok{ }\NormalTok{artigos_aceitos}
\KeywordTok{plot}\NormalTok{(anos, numero_de_artigos_aceitos)}
\end{Highlighting}
\end{Shaded}

\includegraphics{relatorio_files/figure-latex/unnamed-chunk-1-1.pdf}

A análise até este momento nos permitiu entender a estrutura dos dados
do currículo lattes e como esta pode se relacionar com as diferentes
bases de grupos de pesquisa, orientação ou demais publicações.


\end{document}
